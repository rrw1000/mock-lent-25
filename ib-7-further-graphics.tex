\documentclass{tripos}
\begin{document}
\begin{question}[MockIA,year=2025,paper=2,question=7,author=nobody]{Further Graphics}
%%\emph{\ldots Awaiting question 7 on Further Graphics}
  \triposset{fullmarks=20}
  \topic{Further Graphics}
  \begin{enumerate}
\item Compute the normal of the parametric surface
  \[
  x = uv\cos2\pi
  \]
  \[
  y = (u+v)\sin 2\pi
  \]
  \[
  z = u\cos \pi
  \]
  \fullmarks{4}

\item What is the implication for importance sampling of working with a very nonconformally mapped model?
  \fullmarks{2}
\item Recall that, for quaternions:
  \[
  \mathbf{q}^t = e^{t \log \mathbf{q}}
  \]
  \[
  \log \mathbf{q} = \frac{\theta}{2}.\mathbf{s}
  \]
  \[
  e^{\mathbf{q}} = \cos \|\mathbf{q}\| + \frac{\mathbf{q}}{\|\mathbf{q}\|}\sin \|\mathbf{q}\|
  \]
  Show that spherical blending ($(\mathbf{q}_2\mathbf{q}_1^*)\mathbf{q}_1$) interpolates two quaternion rotations. \fullmarks{4}

\item Hence, or otherwise, show that a convex mesh attached to a spherical blend of the rotations of two connected bones cannot become concave by the rotation of the bones.
  \fullmarks{5}

\item For a given scene, assume the rendering equation can be reduced to $L_0(\mathbf{x}, \vec{w}_0) = c \int_{H^2} V(\mathbf{x}, \vec{w}_i)cos \theta_i d\vec{w}_i$ at points $\mathbf{x}$ where $V$ is the binary visibility function.
  \begin{enumerate}
  \item What are the properties of the scene that make this reduction possible? \fullmarks{3}
  \item Assuming a single object and ignoring inter-reflections for that object, what
    is the gradient of $L_0(\mathbf{x}, \vec{w}_0)$ with respect to $x$? \fullmarks{2}
  \end{enumerate}  
\end{enumerate}

%%\typeout{Awaiting question 7 on Further Graphics}
\end{question}
\end{document}
