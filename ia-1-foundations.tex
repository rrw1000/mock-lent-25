\documentclass{tripos}
\begin{document}
\begin{question}[MockIA,year=2025,paper=1,question=1,author=rrw]{Foundations Of Computer Science}
\triposset{fullmarks=20}
\topic{Foundations of Computer Science}

We can define a game in Ocaml with a pair of type definitions like:

\begin{verbatim}
type 'a movelist = NextMove of 'a * (unit -> 'a movelist)
  | MoveNone;;
type ('a, 'b) game = ('a -> 'b -> 'b movelist) * ('a -> 'b -> bool);;
\end{verbatim}

The first function in \verb|game| takes a player and a game state and produces a functional list of next moves.
The second takes a player and game state and tells you if they won or not.

There is a game called nim which, for these purposes will be played like this:

\begin{itemize}
\item There are a number of piles each containing an integer number of items.
\item In each round, each player takes an item from the pile of their choice.
\item The player who takes the last item from any pile (reducing that pile to 0 items) wins the game.
\end{itemize}

\begin{enumerate}
\item Write a function \verb|winner:'a -> 'b -> bool| which tells you if a player can win the game from a state. \fullmarks{3}
\item Write a function \verb|moves:'a -> 'b -> 'b movelist| which constructs a set of valid moves for a player. \fullmarks{4}
\item Write a function \verb|search game players state| which finds all possible wins. \fullmarks{8}
\item Now write a function \verb|best game players state| which finds the first possible winner [hint: try not to be clever - the obvious strategy is fine]. \fullmarks{4}
\item Estimate the big-O complexity of your answer to part (d) \fullmarks{1}
\end{enumerate}

\eject
\end{question}
\end{document}
